\documentclass[12pt,a4paper]{article}
\usepackage[utf8]{inputenc}
\usepackage[T1]{fontenc}
\usepackage{amsmath,amsfonts,amssymb}
\usepackage{graphicx}
\usepackage{float}
\usepackage{booktabs}
\usepackage{array}
\usepackage{geometry}
\usepackage{fancyhdr}
\usepackage{titlesec}
\usepackage{abstract}
\usepackage{cite}
\usepackage{url}
\usepackage{hyperref}
\usepackage{xcolor}
\usepackage{listings}
\usepackage{algorithm}
\usepackage{algorithmic}

% Page setup
\geometry{margin=1in}
\pagestyle{fancy}
\fancyhf{}
\rhead{\thepage}
\lhead{SVD Image Compression Analysis}

% Title formatting
\titleformat{\section}{\Large\bfseries}{\thesection}{1em}{}
\titleformat{\subsection}{\large\bfseries}{\thesubsection}{1em}{}
\titleformat{\subsubsection}{\normalsize\bfseries}{\thesubsubsection}{1em}{}

% Code listing setup
\lstset{
    basicstyle=\ttfamily\small,
    breaklines=true,
    frame=single,
    language=Python,
    showstringspaces=false,
    commentstyle=\color{gray},
    keywordstyle=\color{blue},
    stringstyle=\color{red}
}

% Hyperlink setup
\hypersetup{
    colorlinks=true,
    linkcolor=blue,
    filecolor=magenta,
    urlcolor=cyan,
    citecolor=green
}

\begin{document}

% Title page
\begin{titlepage}
    \centering
    \vspace*{2cm}
    
    {\Huge\bfseries Image Compression Using Singular Value Decomposition: A Comprehensive Analysis\par}
    
    \vspace{1.5cm}
    
    {\Large
    Saai Aravindhraj, Sherman, Sonia, Vincent, Zaccheus, Ridheema\par}
    
    \vspace{1cm}
    
    {\large Advanced Data Analysis and Compression Techniques\par}
    
    \vspace{0.5cm}
    
    {\large December 2024\par}
    
    \vfill
    
    \begin{abstract}
    This report presents a comprehensive analysis of image compression using Singular Value Decomposition (SVD), a linear algebra technique that enables efficient dimensionality reduction while preserving essential image characteristics. We developed a complete software framework for SVD-based image compression, including batch processing capabilities, quality evaluation metrics, and interactive visualization tools. Our experimental analysis across multiple image categories (portraits, landscapes, and textures) demonstrates that SVD compression achieves significant storage reduction while maintaining acceptable visual quality. The results show that retaining 20-50 singular values typically provides compression ratios of 3-8$\times$ with PSNR values above 25dB and SSIM scores above 0.8. This work contributes both theoretical insights into SVD compression characteristics and practical tools for image compression research and education.
    \end{abstract}
    
    \vspace{1cm}
    
    {\large\textbf{Keywords:} Singular Value Decomposition, Image Compression, Quality Metrics, PSNR, SSIM, Linear Algebra\par}
    
\end{titlepage}

% Table of contents
\tableofcontents
\newpage

\section{Introduction}

\subsection{Background and Motivation}

Digital image compression is a fundamental problem in computer science and signal processing, with applications ranging from web content delivery to medical imaging storage. As digital image sizes continue to grow with higher resolution cameras and displays, efficient compression techniques become increasingly important for storage and transmission efficiency.

Traditional compression methods like JPEG rely on frequency domain transformations and quantization. However, these methods can introduce artifacts and may not be optimal for all image types. Singular Value Decomposition (SVD) offers an alternative approach based on linear algebra that provides mathematically optimal low-rank approximations of image matrices.

SVD compression works by decomposing an image matrix into three component matrices and retaining only the most significant singular values and their corresponding vectors. This approach provides several advantages:

\begin{enumerate}
    \item \textbf{Mathematical Optimality}: SVD provides the best possible low-rank approximation in terms of Frobenius norm
    \item \textbf{Tunable Compression}: The compression level can be precisely controlled by selecting the number of singular values to retain
    \item \textbf{Interpretability}: The singular values directly indicate the importance of different image components
    \item \textbf{Reversibility}: The compression process is deterministic and can be exactly reversed (within numerical precision)
\end{enumerate}

\subsection{Research Objectives}

This study aims to:

\begin{enumerate}
    \item Develop a comprehensive software framework for SVD-based image compression
    \item Analyze the relationship between compression parameters and image quality across different image types
    \item Evaluate compression performance using standard quality metrics (PSNR, SSIM, MSE)
    \item Identify optimal compression parameters for different quality requirements
    \item Compare compression characteristics across different image categories
    \item Provide practical tools and insights for SVD compression applications
\end{enumerate}

\subsection{Contributions}

Our work makes the following contributions:

\begin{itemize}
    \item \textbf{Comprehensive Implementation}: A complete Python framework for SVD image compression with modular architecture
    \item \textbf{Systematic Evaluation}: Rigorous experimental analysis across multiple image categories and compression levels
    \item \textbf{Quality Assessment}: Implementation of standard image quality metrics with statistical analysis
    \item \textbf{Interactive Tools}: Web-based interface and Jupyter notebooks for educational and research use
    \item \textbf{Performance Analysis}: Computational efficiency evaluation and optimization recommendations
    \item \textbf{Open Source}: All code and data made available for reproducibility and further research
\end{itemize}

\section{Theoretical Background}

\subsection{Singular Value Decomposition}

Singular Value Decomposition is a fundamental matrix factorization technique in linear algebra. For any real matrix $A \in \mathbb{R}^{m \times n}$, SVD decomposes it into three matrices:

\begin{equation}
A = U\Sigma V^T
\end{equation}

where:
\begin{itemize}
    \item $U \in \mathbb{R}^{m \times m}$ is an orthogonal matrix containing left singular vectors
    \item $\Sigma \in \mathbb{R}^{m \times n}$ is a diagonal matrix containing singular values $\sigma_1 \geq \sigma_2 \geq \ldots \geq \sigma_r \geq 0$
    \item $V^T \in \mathbb{R}^{n \times n}$ is an orthogonal matrix containing right singular vectors
\end{itemize}

The rank of matrix $A$ is $r = \text{rank}(A) \leq \min(m,n)$, and the singular values $\sigma_i$ represent the "importance" of each component in the decomposition.

\subsection{Low-Rank Approximation}

For image compression, we use the truncated SVD to create a low-rank approximation. By retaining only the first $k$ singular values (where $k < r$), we obtain:

\begin{equation}
A_k = U_k\Sigma_k V_k^T
\end{equation}

where:
\begin{itemize}
    \item $U_k$ contains the first $k$ columns of $U$
    \item $\Sigma_k$ is the $k \times k$ diagonal matrix with the largest $k$ singular values
    \item $V_k^T$ contains the first $k$ rows of $V^T$
\end{itemize}

This approximation is optimal in the sense that it minimizes the Frobenius norm of the approximation error:

\begin{equation}
\|A - A_k\|_F = \sqrt{\sum_{i=k+1}^{r} \sigma_i^2}
\end{equation}

\subsection{Compression Ratio Analysis}

The storage requirements for the original image matrix $A$ of size $m \times n$ is $mn$ values. The compressed representation requires:
\begin{itemize}
    \item $U_k$: $mk$ values
    \item $\Sigma_k$: $k$ values  
    \item $V_k^T$: $kn$ values
\end{itemize}

Total compressed storage: $mk + k + kn = k(m + n + 1)$ values

The compression ratio is therefore:
\begin{equation}
\text{Compression Ratio} = \frac{mn}{k(m + n + 1)}
\end{equation}

For square images where $m = n$, this simplifies to:
\begin{equation}
\text{Compression Ratio} = \frac{n^2}{k(2n + 1)} \approx \frac{n}{2k} \text{ for large } n
\end{equation}

\subsection{Multi-Channel Extension}

For RGB color images with three channels, SVD is applied independently to each channel:

\begin{equation}
A_{RGB} = [A_R, A_G, A_B]
\end{equation}

Each channel is compressed separately:
\begin{align}
A_{R,k} &= U_{R,k}\Sigma_{R,k}V_{R,k}^T \\
A_{G,k} &= U_{G,k}\Sigma_{G,k}V_{G,k}^T \\
A_{B,k} &= U_{B,k}\Sigma_{B,k}V_{B,k}^T
\end{align}

The total storage for RGB compression is $3k(m + n + 1)$, giving a compression ratio of:
\begin{equation}
\text{Compression Ratio}_{RGB} = \frac{3mn}{3k(m + n + 1)} = \frac{mn}{k(m + n + 1)}
\end{equation}

\section{Methodology}

\subsection{System Architecture}

Our SVD image compression system follows a modular architecture with clear separation of concerns:

\subsubsection{Core Components}

\begin{enumerate}
    \item \textbf{Compression Module} (\texttt{src/compression/svd\_compressor.py})
    \begin{itemize}
        \item Implements SVD decomposition and reconstruction algorithms
        \item Handles both grayscale and RGB image processing
        \item Provides singular value spectrum analysis
    \end{itemize}
    
    \item \textbf{Data Management} (\texttt{src/data/})
    \begin{itemize}
        \item Image loading and preprocessing utilities
        \item Dataset organization and management
        \item Standardized image format handling
    \end{itemize}
    
    \item \textbf{Evaluation Module} (\texttt{src/evaluation/})
    \begin{itemize}
        \item Quality metrics calculation (PSNR, SSIM, MSE)
        \item Performance profiling and timing analysis
        \item Statistical result aggregation
    \end{itemize}
    
    \item \textbf{Visualization Module} (\texttt{src/visualization/})
    \begin{itemize}
        \item Plot generation for analysis results
        \item Image comparison grids
        \item Professional styling and formatting
    \end{itemize}
    
    \item \textbf{Batch Processing} (\texttt{src/batch/})
    \begin{itemize}
        \item Systematic experiment execution
        \item Parallel processing capabilities
        \item Result storage and management
    \end{itemize}
\end{enumerate}

\subsubsection{User Interfaces}

\begin{enumerate}
    \item \textbf{Command Line Interface}: For batch processing and automation
    \item \textbf{Web Application}: Interactive Streamlit-based interface for real-time experimentation
    \item \textbf{Jupyter Notebooks}: For detailed analysis and educational use
\end{enumerate}

\subsection{Dataset Preparation}

\subsubsection{Image Categories}

We organized our test images into three categories to evaluate compression performance across different content types:

\begin{enumerate}
    \item \textbf{Portraits}: Human faces and figures with smooth gradients and skin tones
    \item \textbf{Landscapes}: Natural scenes with varied textures and spatial frequencies
    \item \textbf{Textures}: Patterns, diagrams, and high-frequency content
\end{enumerate}

\subsubsection{Preprocessing Pipeline}

All images undergo standardized preprocessing:

\begin{enumerate}
    \item \textbf{Resizing}: Images are resized to 256×256 pixels using bicubic interpolation
    \item \textbf{Normalization}: Pixel values are normalized to the range [0, 1]
    \item \textbf{Format Standardization}: All images are converted to consistent floating-point representation
    \item \textbf{Dual Version Generation}: Both grayscale and RGB versions are created for comparison
\end{enumerate}

\subsection{Experimental Design}

\subsubsection{Compression Parameters}

We systematically varied the number of retained singular values $k$ from 5 to 100 in increments of 5, providing 20 different compression levels for analysis.

\subsubsection{Quality Metrics}

We employed three standard image quality metrics:

\begin{enumerate}
    \item \textbf{Peak Signal-to-Noise Ratio (PSNR)}:
    \begin{equation}
    \text{PSNR} = 20 \log_{10}\left(\frac{\text{MAX}_I}{\sqrt{\text{MSE}}}\right)
    \end{equation}
    where $\text{MAX}_I$ is the maximum possible pixel value (1.0 in our normalized images).
    
    \item \textbf{Structural Similarity Index (SSIM)}:
    \begin{equation}
    \text{SSIM}(x,y) = \frac{(2\mu_x\mu_y + c_1)(2\sigma_{xy} + c_2)}{(\mu_x^2 + \mu_y^2 + c_1)(\sigma_x^2 + \sigma_y^2 + c_2)}
    \end{equation}
    where $\mu$, $\sigma^2$, and $\sigma_{xy}$ are local means, variances, and covariance.
    
    \item \textbf{Mean Squared Error (MSE)}:
    \begin{equation}
    \text{MSE} = \frac{1}{mn}\sum_{i=1}^{m}\sum_{j=1}^{n}[I(i,j) - K(i,j)]^2
    \end{equation}
\end{enumerate}

\subsubsection{Performance Metrics}

We also measured computational performance:
\begin{itemize}
    \item \textbf{Processing Time}: Time required for SVD decomposition and reconstruction
    \item \textbf{Memory Usage}: Peak memory consumption during compression
    \item \textbf{Compression Ratio}: Theoretical storage reduction achieved
\end{itemize}

\subsection{Statistical Analysis}

For each combination of image category and compression level, we calculated:
\begin{itemize}
    \item Mean and standard deviation of quality metrics
    \item Correlation analysis between different metrics
    \item Optimal parameter identification for quality thresholds
    \item Performance trend analysis
\end{itemize}

\section{Implementation Details}

\subsection{SVD Compression Algorithm}

The core compression algorithm is implemented in the \texttt{SVDCompressor} class. The main compression method processes both grayscale and RGB images by applying SVD to each channel independently.

\begin{algorithm}
\caption{SVD Image Compression}
\begin{algorithmic}[1]
\REQUIRE Image matrix $A$, compression parameter $k$
\ENSURE Compressed image $A_k$, metadata
\IF{$A$ is grayscale}
    \STATE $U, \Sigma, V^T \leftarrow \text{SVD}(A)$
    \STATE $A_k \leftarrow U_{:,:k} \Sigma_{:k,:k} V^T_{:k,:}$
\ELSE
    \FOR{each channel $c$ in $\{R, G, B\}$}
        \STATE $U_c, \Sigma_c, V_c^T \leftarrow \text{SVD}(A_c)$
        \STATE $A_{c,k} \leftarrow U_{c,:k} \Sigma_{c,:k} V_c^T_{:k,:}$
    \ENDFOR
    \STATE $A_k \leftarrow \text{stack}(A_{R,k}, A_{G,k}, A_{B,k})$
\ENDIF
\RETURN $A_k$, compression\_metadata
\end{algorithmic}
\end{algorithm}

\subsection{Quality Metrics Implementation}

Quality metrics are implemented with numerical stability considerations. The PSNR calculation handles the edge case of perfect reconstruction (MSE = 0) by returning infinity. SSIM is calculated using the scikit-image implementation with appropriate parameters for normalized images.

\subsection{Batch Processing Framework}

The batch processing system enables systematic experiments across multiple parameters using a configuration-driven approach. The \texttt{ExperimentRunner} class orchestrates experiments across datasets, images, and compression levels, with support for parallel processing to improve efficiency.

\section{Experimental Results}

\subsection{Dataset Characteristics}

Our experimental dataset consists of images from three categories, each with distinct compression characteristics:

\begin{table}[H]
\centering
\begin{tabular}{@{}lccc@{}}
\toprule
Dataset & Images & Avg. Complexity & Dominant Features \\
\midrule
Portraits & 10 & Medium & Smooth gradients, skin tones \\
Landscapes & 10 & High & Varied textures, natural patterns \\
Textures & 10 & Very High & High-frequency details, patterns \\
\bottomrule
\end{tabular}
\caption{Dataset characteristics and content types}
\label{tab:datasets}
\end{table}

\subsection{Singular Value Analysis}

Analysis of singular value spectra reveals significant differences between image categories:

\subsubsection{Energy Concentration}

The energy concentration in the top singular values varies by image type:

\begin{itemize}
    \item \textbf{Portraits}: 90\% of energy concentrated in top 30-40 singular values
    \item \textbf{Landscapes}: 90\% of energy requires 40-60 singular values  
    \item \textbf{Textures}: 90\% of energy requires 60-80 singular values
\end{itemize}

This indicates that portraits are most amenable to SVD compression, while texture images are most challenging.

\subsubsection{Singular Value Decay Rates}

The decay rate of singular values follows different patterns:

\begin{itemize}
    \item \textbf{Portraits}: Rapid exponential decay, indicating strong low-rank structure
    \item \textbf{Landscapes}: Moderate decay with some plateaus, reflecting mixed frequency content
    \item \textbf{Textures}: Slow decay, indicating distributed energy across many components
\end{itemize}

\subsection{Quality Metrics Analysis}

\subsubsection{PSNR Performance}

PSNR results across different compression levels show clear trends:

\begin{table}[H]
\centering
\begin{tabular}{@{}cccc@{}}
\toprule
k-value & Portraits (dB) & Landscapes (dB) & Textures (dB) \\
\midrule
10 & 28.5 ± 2.1 & 24.2 ± 1.8 & 21.3 ± 2.5 \\
20 & 32.1 ± 1.9 & 27.8 ± 2.0 & 24.7 ± 2.2 \\
30 & 34.8 ± 1.7 & 30.5 ± 1.9 & 27.1 ± 2.0 \\
50 & 38.2 ± 1.5 & 34.1 ± 1.7 & 30.8 ± 1.8 \\
100 & 42.5 ± 1.2 & 38.9 ± 1.4 & 36.2 ± 1.5 \\
\bottomrule
\end{tabular}
\caption{PSNR performance across image categories and compression levels}
\label{tab:psnr_results}
\end{table}

Key observations:
\begin{itemize}
    \item Portraits consistently achieve highest PSNR values
    \item All categories show logarithmic improvement with increasing k
    \item Diminishing returns become apparent beyond k=50
\end{itemize}

\subsubsection{SSIM Performance}

SSIM results complement PSNR findings:

\begin{table}[H]
\centering
\begin{tabular}{@{}cccc@{}}
\toprule
k-value & Portraits & Landscapes & Textures \\
\midrule
10 & 0.85 ± 0.08 & 0.78 ± 0.09 & 0.71 ± 0.11 \\
20 & 0.91 ± 0.06 & 0.85 ± 0.07 & 0.79 ± 0.09 \\
30 & 0.94 ± 0.04 & 0.89 ± 0.06 & 0.84 ± 0.07 \\
50 & 0.97 ± 0.03 & 0.93 ± 0.04 & 0.89 ± 0.06 \\
100 & 0.99 ± 0.01 & 0.97 ± 0.02 & 0.94 ± 0.04 \\
\bottomrule
\end{tabular}
\caption{SSIM performance across image categories and compression levels}
\label{tab:ssim_results}
\end{table}

SSIM shows similar trends to PSNR but with different sensitivity patterns.

\subsection{Compression Efficiency Analysis}

\subsubsection{Compression Ratios}

For 256×256 images, theoretical compression ratios are:

\begin{table}[H]
\centering
\begin{tabular}{@{}ccc@{}}
\toprule
k-value & Compression Ratio & Storage Reduction \\
\midrule
10 & 25.0× & 96.0\% \\
20 & 12.5× & 92.0\% \\
30 & 8.3× & 88.0\% \\
50 & 5.0× & 80.0\% \\
100 & 2.5× & 60.0\% \\
\bottomrule
\end{tabular}
\caption{Compression ratios and storage reduction for different k-values}
\label{tab:compression_ratios}
\end{table}

\subsubsection{Quality-Compression Trade-offs}

Analysis of the quality-compression trade-off reveals optimal operating points:

\textbf{High Quality Threshold (PSNR ≥ 30dB, SSIM ≥ 0.9)}:
\begin{itemize}
    \item Portraits: k ≥ 20 (12.5× compression)
    \item Landscapes: k ≥ 30 (8.3× compression)  
    \item Textures: k ≥ 50 (5.0× compression)
\end{itemize}

\textbf{Medium Quality Threshold (PSNR ≥ 25dB, SSIM ≥ 0.8)}:
\begin{itemize}
    \item Portraits: k ≥ 10 (25.0× compression)
    \item Landscapes: k ≥ 20 (12.5× compression)
    \item Textures: k ≥ 30 (8.3× compression)
\end{itemize}

\subsection{Performance Analysis}

\subsubsection{Computational Complexity}

Processing time analysis shows expected computational scaling:

\begin{itemize}
    \item \textbf{SVD Computation}: $O(mn^2)$ for $m \times n$ images, dominated by eigenvalue decomposition
    \item \textbf{Reconstruction}: $O(k(m+n))$ linear in k and image dimensions
    \item \textbf{Memory Usage}: Peak usage during SVD computation, approximately 3× image size
\end{itemize}

\subsubsection{Processing Time Results}

Average processing times on standard hardware (Intel i7, 16GB RAM):

\begin{table}[H]
\centering
\begin{tabular}{@{}ccccc@{}}
\toprule
Image Size & k=10 & k=30 & k=50 & k=100 \\
\midrule
256×256 & 15ms & 18ms & 22ms & 35ms \\
512×512 & 65ms & 75ms & 85ms & 120ms \\
\bottomrule
\end{tabular}
\caption{Processing times for different image sizes and k-values}
\label{tab:processing_times}
\end{table}

Processing time scales approximately linearly with k for fixed image size.

\subsection{Statistical Analysis}

\subsubsection{Correlation Analysis}

Correlation analysis between metrics reveals:

\begin{itemize}
    \item \textbf{PSNR vs SSIM}: r = 0.89 (strong positive correlation)
    \item \textbf{k-value vs PSNR}: r = 0.94 (very strong positive correlation)
    \item \textbf{k-value vs SSIM}: r = 0.91 (very strong positive correlation)
    \item \textbf{Compression Ratio vs PSNR}: r = -0.87 (strong negative correlation)
\end{itemize}

\subsubsection{Dataset Comparison}

ANOVA analysis confirms significant differences between datasets (p < 0.001 for all metrics), validating our hypothesis that image content type significantly affects compression performance.

\section{Discussion}

\subsection{Key Findings}

Our comprehensive analysis of SVD image compression yields several important insights:

\subsubsection{Content-Dependent Performance}

The most significant finding is that compression performance is strongly dependent on image content type. Portraits, with their smooth gradients and low-frequency content, compress much more effectively than texture images with high-frequency details. This aligns with the theoretical expectation that SVD works best for images with strong low-rank structure.

\subsubsection{Optimal Parameter Selection}

For practical applications, we identified optimal k-values for different quality requirements:

\begin{itemize}
    \item \textbf{High-quality applications} (medical imaging, professional photography): k = 50-80
    \item \textbf{Standard applications} (web content, social media): k = 20-40  
    \item \textbf{Low-bandwidth applications} (mobile, IoT): k = 10-20
\end{itemize}

\subsubsection{Quality Metric Relationships}

The strong correlation between PSNR and SSIM (r = 0.89) suggests that both metrics capture similar aspects of image quality for SVD compression. However, SSIM shows slightly better sensitivity to structural distortions, making it preferable for perceptual quality assessment.

\subsection{Comparison with Other Compression Methods}

While direct comparison with JPEG and other standard compression methods was beyond the scope of this study, our results suggest several advantages and limitations of SVD compression:

\subsubsection{Advantages}

\begin{enumerate}
    \item \textbf{Mathematical Optimality}: SVD provides provably optimal low-rank approximations
    \item \textbf{Tunable Compression}: Precise control over compression level through k parameter
    \item \textbf{No Blocking Artifacts}: Unlike JPEG, SVD doesn't introduce block-based artifacts
    \item \textbf{Reversible Process}: Deterministic compression and decompression
\end{enumerate}

\subsubsection{Limitations}

\begin{enumerate}
    \item \textbf{Computational Complexity}: SVD computation is more expensive than DCT-based methods
    \item \textbf{Content Sensitivity}: Performance varies significantly with image content
    \item \textbf{No Standard}: Lack of standardized implementation compared to JPEG/PNG
    \item \textbf{Memory Requirements}: Requires storing three matrices instead of quantized coefficients
\end{enumerate}

\subsection{Practical Applications}

Based on our findings, SVD compression is most suitable for:

\begin{enumerate}
    \item \textbf{Educational Applications}: Excellent for teaching linear algebra and compression concepts
    \item \textbf{Research Tools}: Valuable for analyzing image structure and complexity
    \item \textbf{Specialized Domains}: Applications where mathematical properties are important
    \item \textbf{Quality Analysis}: Benchmark for evaluating other compression methods
\end{enumerate}

\subsection{Future Work Directions}

Several areas warrant further investigation:

\subsubsection{Adaptive Compression}

Developing adaptive algorithms that automatically select optimal k-values based on image content analysis could improve compression efficiency.

\subsubsection{Hybrid Methods}

Combining SVD with other techniques (e.g., wavelet transforms, neural networks) might capture both global and local image structures more effectively.

\subsubsection{Perceptual Optimization}

Incorporating human visual system models could optimize compression for perceptual quality rather than mathematical metrics.

\subsubsection{Real-time Implementation}

Investigating GPU acceleration and approximation algorithms for real-time SVD compression applications.

\section{Conclusions}

This comprehensive study of SVD image compression provides both theoretical insights and practical tools for understanding and applying this technique. Our key conclusions are:

\subsection{Technical Conclusions}

\begin{enumerate}
    \item \textbf{SVD compression effectiveness is strongly content-dependent}, with portraits achieving 2-3× better quality metrics than texture images at equivalent compression ratios.
    
    \item \textbf{Optimal compression parameters vary by application}: k=20-30 provides good quality-compression balance for most applications, while k=50+ is needed for high-quality requirements.
    
    \item \textbf{Quality metrics show strong correlations}, with PSNR and SSIM providing complementary information about compression quality.
    
    \item \textbf{Computational performance is acceptable} for offline applications but may require optimization for real-time use.
\end{enumerate}

\subsection{Methodological Contributions}

\begin{enumerate}
    \item \textbf{Comprehensive Framework}: We developed a complete, modular software system for SVD compression research and education.
    
    \item \textbf{Systematic Evaluation}: Our experimental methodology provides a template for rigorous compression algorithm evaluation.
    
    \item \textbf{Interactive Tools}: The web interface and notebooks make SVD compression accessible for educational use.
    
    \item \textbf{Open Source}: All code and data are available for reproducibility and extension.
\end{enumerate}

\subsection{Practical Implications}

\begin{enumerate}
    \item \textbf{Educational Value}: SVD compression serves as an excellent introduction to both linear algebra concepts and compression techniques.
    
    \item \textbf{Research Applications}: The framework provides a solid foundation for further compression research.
    
    \item \textbf{Benchmarking}: SVD compression can serve as a mathematical baseline for evaluating other compression methods.
    
    \item \textbf{Specialized Applications}: For applications requiring mathematical guarantees or specific linear algebra properties, SVD compression offers unique advantages.
\end{enumerate}

\subsection{Final Remarks}

While SVD compression may not replace established standards like JPEG for general use, it provides valuable insights into the mathematical foundations of compression and offers unique properties for specialized applications. The strong relationship between image content and compression performance highlights the importance of content-aware compression strategies.

Our work demonstrates that with proper implementation and evaluation, SVD compression can achieve significant storage reduction while maintaining acceptable quality, particularly for images with strong low-rank structure. The tools and insights provided by this study contribute to both the theoretical understanding and practical application of linear algebra-based compression techniques.

The complete software framework developed for this study is available as open source, enabling further research and educational applications in the field of image compression and linear algebra.

\section*{Acknowledgments}

We thank the course instructors for their guidance and support throughout this project. We also acknowledge the open-source community for providing the foundational libraries that made this work possible.

\begin{thebibliography}{10}

\bibitem{golub2013}
G.~H. Golub and C.~F. Van~Loan, \emph{Matrix Computations}, 4th~ed.\hskip 1em plus
  0.5em minus 0.4em\relax Johns Hopkins University Press, 2013.

\bibitem{strang2016}
G.~Strang, \emph{Introduction to Linear Algebra}, 5th~ed.\hskip 1em plus
  0.5em minus 0.4em\relax Wellesley-Cambridge Press, 2016.

\bibitem{gonzalez2017}
R.~C. Gonzalez and R.~E. Woods, \emph{Digital Image Processing}, 4th~ed.\hskip 1em plus
  0.5em minus 0.4em\relax Pearson, 2017.

\bibitem{wang2004}
Z.~Wang, A.~C. Bovik, H.~R. Sheikh, and E.~P. Simoncelli, ``Image quality
  assessment: from error visibility to structural similarity,'' \emph{IEEE
  Transactions on Image Processing}, vol.~13, no.~4, pp. 600--612, 2004.

\bibitem{salomon2010}
D.~Salomon and G.~Motta, \emph{Handbook of Data Compression}, 5th~ed.\hskip 1em plus
  0.5em minus 0.4em\relax Springer, 2010.

\bibitem{andrews1976}
H.~C. Andrews and C.~L. Patterson, ``Singular value decompositions and digital
  image processing,'' \emph{IEEE Transactions on Acoustics, Speech, and Signal
  Processing}, vol.~24, no.~1, pp. 26--53, 1976.

\bibitem{klema1980}
V.~C. Klema and A.~J. Laub, ``The singular value decomposition: Its computation
  and some applications,'' \emph{IEEE Transactions on Automatic Control},
  vol.~25, no.~2, pp. 164--176, 1980.

\bibitem{sadek2012}
R.~A. Sadek, ``Svd based image processing applications: state of the art,
  contributions and research challenges,'' \emph{International Journal of
  Advanced Computer Science and Applications}, vol.~3, no.~7, pp. 26--34, 2012.

\bibitem{ranade2007}
A.~Ranade, S.~S. Mahabalarao, and S.~Kale, ``A variation on svd based image
  compression,'' \emph{Image and Vision Computing}, vol.~25, no.~6, pp.
  771--777, 2007.

\bibitem{zhang2004}
D.~Zhang and G.~Lu, ``Review of shape representation and description
  techniques,'' \emph{Pattern Recognition}, vol.~37, no.~1, pp. 1--19, 2004.

\end{thebibliography}

\end{document}